% Created 2019-12-16 Mon 12:41
\documentclass[12pt,a4paper]{article}
\usepackage[utf8]{inputenc}
\usepackage[T1]{fontenc}
\usepackage{fixltx2e}
\usepackage{graphicx}
\usepackage{longtable}
\usepackage{float}
\usepackage{wrapfig}
\usepackage{rotating}
\usepackage[normalem]{ulem}
\usepackage{amsmath}
\usepackage{textcomp}
\usepackage{marvosym}
\usepackage{wasysym}
\usepackage{amssymb}
\usepackage{hyperref}
\tolerance=1000
\usepackage[margin=3cm]{geometry} 	   % Choose your margin here.
\usepackage{tikz,pgfplots}
\usetikzlibrary{calc,patterns,arrows,decorations.pathmorphing,decorations.markings}
\usepackage{array,makecell,multirow}
\pgfplotsset{width=16cm,height=6cm, compat=1.8}
\usepackage{amsmath,mathtools,amssymb,mathrsfs}
\usepackage{cancel}
\setcounter{secnumdepth}{1}
\author{nybo}
\date{\today}
\title{readme}
\hypersetup{
  pdfkeywords={},
  pdfsubject={},
  pdfcreator={Emacs 25.2.2 (Org mode 8.2.10)}}
\begin{document}

\maketitle
\tableofcontents

Personal note to help me remember 

\section{Norms}
\label{sec-1}
\subsection*{def}
\label{sec-1-1}
\begin{enumerate}
\item Positivity  $\|x\| \geq 0$
\item Positive definiteness $\|x\|=0 \Longleftrightarrow x=0$
\item Homogeneity $\|\alpha x\|=|\alpha|\|x\|$ for arbitrary scalar $\alpha$
\item Triangle inequality $\|x+y\| \leq\|x\|+\|y\|$
\end{enumerate}
Note: not sure if this holds for ever norm
\subsection*{The different matrix ones (Norms on $A \in \mathbb{R}^{m \times n}$)}
\label{sec-1-2}
$$
\|A\|_{1}=\max _{1 \leq j \leq n} \sum_{i=1}^{m}\left|a_{i j}\right|
$$

$$
\|A\|_{2}=\sqrt{\lambda_{\max }\left(A^{T} A\right)}
$$

Frobenius norm, sometimes also called the Euclidean norm

$$
\|\mathrm{A}\|_{F} \equiv \sqrt{\sum_{i=1}^{m} \sum_{j=1}^{n}\left|a_{i j}\right|^{2}}
$$

$$
\|A\|_{\infty}=\max _{1 \leq i \leq m} \sum_{j=1}^{n}\left|a_{i j}\right|
$$
\subsection*{Vector norms (Norms on $\mathbb{R}^{n}$)}
\label{sec-1-3}

$$|\mathbf{x}|_{p} \equiv\left(\sum_{i}\left|x_{i}\right|^{p}\right)^{1 / p}$$

special case:

$$|\mathbf{x}|_{\infty} \equiv \max _{i}\left|x_{i}\right|$$



Tips: pretty sure think $\|\cdot\|$ usually just refers to the 2-norm.


\subsection*{Scalar norm (Norm on $C[a, b]$)}
\label{sec-1-4}

\begin{equation}
\left.\begin{array}{l}
{\|f\|_{p}=\left(\int_{a}^{b}|f(\tau)|^{p} d \tau\right)^{\frac{1}{p}}, \quad p \in[1, \infty]} \\ 
{\|f\|_{\infty}=\displaystyle\sup _{\scriptscriptstyle a \leq t \leq b}|f(t)| }\end{array} \quad,
\right\} \quad \mathscr{L}_{p}-\text { norms }
\end{equation}

$$
C[0, \infty), \mathscr{L}_{p} \text{ is a Banach space}
$$

$f \in \mathscr{L}_{p} \Leftrightarrow\|f\|_{p}$ is bounded, i.e. $\exists c:\|f\|_{p} \leq c$
\section{Tensor rank}
\label{sec-2}
\begin{table}[h]
\begin{tabular}{cl}
rank                & object   \\
\hline
0                   & scalar   \\
1                   & vector  \\
2                   & matrix (/Dyad)  \\
$\geq 3$ & tensor        
\end{tabular}
\end{table}
Also sometimes  triad, tetrad are used to refer to tensors of 
rank 3 and 4 respectivly. Some refer to the rank of a tensor as
its order or its degree.




\section{SVD}
\label{sec-3}
Example: 
$A=\left[\begin{array}{lll}{0} & {1} & {1} \\ {\sqrt{2}} & {2} & {0} \\ {0} & {1} & {1}\end{array}\right]$

The SVD is defined as
$$
A=P \Sigma Q^{T}
$$

\subsection*{Method}
\label{sec-3-1}
Computing:
$$
A A^{T}=\left[\begin{array}{lll}{2} & {2} & {2} \\ {2} & {6} & {2} \\ {2} & {2} & {2}\end{array}\right]
$$

\begin{equation}
\begin{aligned}
-\lambda^{3}+10 \lambda^{2}-16 \lambda &=-\lambda\left(\lambda^{2}-10 \lambda+16\right) \\
 &=-\lambda(\lambda-8)(\lambda-2) 
\end{aligned}
\end{equation}


Eigenvals of $A A^{T}$ are $\lambda=8, \lambda=2, \lambda=0$, thus the singular values
 are $\sigma_{1}=2 \sqrt{2}, \sigma_{2}=\sqrt{2}\left(\text { and } \sigma_{3}=0\right)$.


Giving out the matrix
$$
\Sigma= 0_{3x3}+\sigma=
\left[\begin{array}{ccc}{2 \sqrt{2}} & {0} & {0} \\ {0} & {\sqrt{2}} & {0} \\ {0} & {0} & {0}\end{array}\right]
$$


Finding the eigenvectors $(A-\lambda I) \mathbf{x}=\mathbf{0}$, we get respetivly to the egienvector
 who is described before: $p_{1}=\left(\frac{1}{\sqrt{6}}, \frac{2}{\sqrt{6}}, \frac{1}{\sqrt{6}}\right)$,
 $p_{2}=\left(-\frac{1}{\sqrt{3}}, \frac{1}{\sqrt{3}},-\frac{1}{\sqrt{3}}\right)$ and 
$p_{3}=\left(\frac{1}{\sqrt{2}}, 0,-\frac{1}{\sqrt{2}}\right)$ (note: normalised vectors).

Yeilding

$$
P= \left[  {p_1}^T {p_2}^T {p_3}^T     \right]
=\left[\begin{array}{ccc}{\frac{1}{\sqrt{6}}} & {-\frac{1}{\sqrt{3}}} & {\frac{1}{\sqrt{2}}} \\ {\frac{2}{\sqrt{6}}} & {\frac{1}{\sqrt{3}}} & {0} \\ {\frac{1}{\sqrt{6}}} & {-\frac{1}{\sqrt{3}}} & {-\frac{1}{\sqrt{2}}}\end{array}\right]
$$




$$
A^{T} A=\left[\begin{array}{ccc}{2} & {2 \sqrt{2}} & {0} \\ {2 \sqrt{2}} & {6} & {2} \\ {0} & {2} & {2}\end{array}\right]
$$
With the eigenvals $\lambda=8, \lambda=2, \lambda=0$ with eigenvectors
$q_{1}=\left(\frac{1}{\sqrt{6}}, \frac{3}{\sqrt{12}}, \frac{1}{\sqrt{12}}\right), q_{2}=\left(\frac{1}{\sqrt{3}}, 0,-\frac{2}{\sqrt{6}}\right) \text { and } q_{3}=\left(\frac{1}{\sqrt{2}},-\frac{1}{2}, \frac{1}{2}\right)$. (Acually
 can also use the the formula $p_{i}=\frac{1}{\sigma_{i}} A^{T} p_{i}$ to get the various $q_i$.


$$
Q=\left[  {q_1}^T {q_2}^T {q_3}^T     \right]
\left[\begin{array}{ccc}{\frac{1}{\sqrt{6}}} & {\frac{1}{\sqrt{3}}} & {\frac{1}{\sqrt{2}}} \\ {\frac{3}{\sqrt{12}}} & {0} & {-\frac{1}{2}} \\ {\frac{1}{\sqrt{12}}} & {-\frac{2}{\sqrt{6}}} & {\frac{1}{2}}\end{array}\right]
$$


We have then the SVD defined as
$$
A=P \Sigma Q^{T}
$$

\section{Covar}
\label{sec-4}


\newpage

\section{change of basis}
\label{sec-5}
\subsection*{eigenvalue  decomposition}
\label{sec-5-1}
$$
A=\left[\begin{array}{lll}{4} & {0} & {1} \\ {2} & {3} & {2} \\ {1} & {0} & {4}\end{array}\right]
$$
\begin{align*}
\lambda&=5 \quad \operatorname{NUL}(A-5 I) =\text { SPAN }\left\{\left[\begin{array}{l}{1} \\ {2} \\ {1}\end{array}\right]\right\} \\
\lambda&=3 \quad \operatorname{NUL}(A-3 I) = \text { SPAN }\left\{\left[\begin{array}{l}{0} \\ {1} \\ {0}\end{array}\right],\left[\begin{array}{c}{-1} \\ {0} \\ {1}\end{array}\right]\right\}
\end{align*}
$$
D=\left[\begin{array}{lll}{5} & {0} & {0} \\ {0} & {3} & {0} \\ {0} & {0} & {3}\end{array}\right]
P=\left[\begin{array}{llc}{1} & {0} & {-1} \\ {2} & {1} & {0} \\ {1} & {0} & {1}\end{array}\right]
$$
OK, good
$$
A=P D P^{-1}
$$
\subsection*{jordan}
\label{sec-5-2}
$$
A=\left[\begin{array}{lll}{1} & {1} & {1} \\ {0} & {1} & {0} \\ {0} & {0} & {1}\end{array}\right]
$$

\begin{align*}
\lambda&=1 \quad \operatorname{NUL}(A-1 I) = \text { SPAN }\left\{\left[\begin{array}{l}{1} \\ {0} \\ {0}\end{array}\right],\left[\begin{array}{c}{0} \\ {1} \\ {-1}\end{array}\right]\right\}
\end{align*}
2 eigenvectors? :-d    (for some reason we are calling these vectors the first and third)

$$
\left[\begin{array}{lll}
{\lambda} & {1} & ?\\ 
{0} & {\lambda}&  ?\\ 
{0} & {0} &? 
\end{array}\right]
$$

\begin{align*}
A V_{1}&=\lambda V_{1} \\
 A V_{2}&=V_{1}+\lambda V_{2}
\end{align*}

\begin{equation*}
\begin{array}{l}{A V_{2}-\lambda V_{2}=V_{1}} \\ {(A-\lambda I) V_{2}=V_{1}}\end{array}
\end{equation*}
$V_1$ is given what is $V_2$? let $V_1=[1,0,0]^T$
\begin{align*}
(A-1 I) V_{2}&=V_{1}\\
\left[\begin{array}{lll}{0} & {1} & {1} \\ {0} & {0} & {0} \\ {0} & {0} & {0}\end{array}\right] V_{2}&=\left[\begin{array}{l}{1} \\ {0} \\ {0}\end{array}\right]
\end{align*}

\begin{equation}
V_{2}=\left[\begin{array}{l}{0} \\ {0} \\ {1}\end{array}\right]+\bcancel{x\left[\begin{array}{l}{1} \\ {0} \\ {0}\end{array}\right]+y\left[\begin{array}{c}{0} \\ {1} \\ {-1}\end{array}\right]}
\end{equation}
Let 
\begin{equation}
V_{2}=\left[\begin{array}{l}{0} \\ {0} \\ {1}\end{array}\right]
\end{equation}

\begin{equation}
V_{3}=\left[\begin{array}{c}{0} \\ {1} \\ {-1}\end{array}\right]
\end{equation}

\begin{equation}
A \quad \text { has the from }\left[\begin{array}{lll}1  &  1 &0 \\  0  &  1 &0 \\  0  &  0 &1\end{array}\right]
\end{equation}

\begin{equation}
P=\left[\begin{array}{lll}{V_{1}} & {V_{2}} & {V_{3}}\end{array}\right]=
 \left[\begin{array}{ccc}{1} & {0} & {0} \\ {0} & {0} & {1} \\ {0} & {1} & {-1}\end{array}\right]
\end{equation}
then
\begin{equation}
A=P J P^{-1} \quad J=\begin{equation}
\left[\begin{array}{lll}{1} & {1} & {0} \\ {0} & {1} & {0} \\ {0} & {0} & {1}\end{array}\right]
\end{equation}
\end{equation}



\section{linear def}
\label{sec-6}

\section{positive definite}
\label{sec-7}
A square matrix A is positive definite if there is a positive scalar α such that
\begin{equation}
x^{T} A x \geq \alpha x^{T} x, \quad \text { for all } x \in \mathbf{R}^{n}
\end{equation}
It is positive semidefinite if
\begin{equation}
x^{T} A x \geq 0, \quad \text { for all } x \in \mathbb{R}^{n}
\end{equation}
We can recognize that a symmetric matrix is positive definite by computing its eigenvalues
and verifying that they are all positive, or by performing a Cholesky factorization. 



\section{rotation things}
\label{sec-8}


3D point as 3-vector
$$
\mathbf{X}=\left[\begin{array}{l}{X} \\ {Y} \\ {Z}\end{array}\right]
$$

3D point using affine homogeneous
$$
\left[\begin{array}{l}{\mathbf{X}} \\ {1}\end{array}\right]=\left[\begin{array}{l}{X} \\ {Y} \\ {Z} \\ {1}\end{array}\right]
$$
Inverse of rotation matrix
\begin{equation}
\begin{aligned} \mathrm{X}^{\prime} &=\mathrm{RX}+\mathrm{t} \\
 \mathrm{X}^{\prime}-\mathrm{t} &=\mathrm{RX} \\ 
\mathrm{R}^{\top}\left(\mathrm{X}^{\prime}-\mathrm{t}\right) &=\mathrm{X} \\
 \mathrm{R}^{\top} \mathrm{X}^{\prime}-\mathrm{R}^{\top} \mathrm{t} &=\mathrm{X} 
\end{aligned}
\end{equation}
This gives in homogenius

$$
\left[\begin{array}{cc}{\mathrm{R}^{\top}} & {-\mathrm{R}^{\top} \mathbf{t}} \\ {0^{\top}} & {1}\end{array}\right]\left[\begin{array}{l}{\mathrm{X}^{\prime}} \\ {1}\end{array}\right]=\left[\begin{array}{l}{\mathrm{X}} \\ {1}\end{array}\right]
$$
\subsection*{Rodrigues rotation}
\label{sec-8-1}
rotation matrix in vector form
\begin{equation} 
\begin{array}{l} \theta \leftarrow norm(r) \\ r  \leftarrow r/ \theta \\ R =  \cos{\theta} I + (1- \cos{\theta} ) r r^T +  \sin{\theta}\left[\begin{array}{ccc}{0} & {-r_{z}} & {r_{y}} \\ {r_{z}} & {0} & {-r_{x}} \\ {-r_{y}} & {r_{x}} & {0}\end{array}\right] \end{array}
\end{equation}

Inverse transformation can be also done easily, since




\begin{equation}
 \sin{\theta}\left[\begin{array}{ccc}{0} & {-r_{z}} & {r_{y}} \\ {r_{z}} & {0} & {-r_{x}} \\ {-r_{y}} & {r_{x}} & {0}\end{array}\right]= \frac{R - R^T}{2}
\end{equation}

A rotation vector is a convenient and most compact representation of a rotation matrix (since any rotation matrix has just 3 degrees of freedom)

\newpage
\section{Logic}
\label{sec-9}
\subsection*{Necessity and sufficiency}
\label{sec-9-1}
\begin{equation*}
\begin{array}{|c|c|c|c|c|}\hline S & {N} & {S \Rightarrow N} & {S \Leftarrow N} & {S \Leftrightarrow N} \\ \hline T & {T} & {T} & {T} & {T} \\ \hline T & {F} & {F} & {T} & {F} \\ \hline F & {T} & {T} & {F} & {F} \\ \hline F & {F} & {T} & {T} & {T} \\ \hline\end{array}
\end{equation*}

\begin{center}
\begin{tikzpicture}
\def\radius{2cm}
\def\mycolorbox#1{\textcolor{#1}{\rule{2ex}{2ex}}}
\colorlet{colori}{blue!70}
\colorlet{colorii}{red!70}

\coordinate (ceni);
\coordinate[xshift=\radius] (cenii);

\draw[fill=colori,fill opacity=0.5] (ceni) circle (\radius);
\draw[fill=colorii,fill opacity=0.5] (cenii) circle (\radius);

\draw  ([xshift=-20pt,yshift=20pt]current bounding box.north west) 
  rectangle ([xshift=20pt,yshift=-20pt]current bounding box.south east);

\node[yshift=10pt] at (current bounding box.north) {Venn diagram };
\node[xshift=-.5\radius] at (ceni) {$\mathbf{S}$};
\node[xshift=.5\radius] at (cenii) {$\mathbf{N}$};
\node[xshift=.9\radius] at (ceni) {$\mathbf{S}\cap\mathbf{N}}$};
\node[xshift=10pt,yshift=10pt] at (current bounding box.south west) {$\emptyset$};
\end{tikzpicture}
\end{center}

\subsection*{Classification: True vs. False and Positive vs. Negative}
\label{sec-9-2}

\subsubsection*{Confusion matrix}
\label{sec-9-2-1}
Example       based off   one story from   Aesop's Fables:
\begin{table}[h!]
\begin{tabular}{|l|l|}
\hline
\makecell[l]{\textbf{True Positive (TP):}\\
Reality: A wolf threatened. \\
Shepherd said: "Wolf." \\
Outcome: Shepherd is a hero.   }         & 
\makecell[l]{\textbf{False Positive (FP):}\\
Reality: No wolf threatened. \\
Shepherd said: "Wolf." \\
Outcome: Villagers are \\ angry at shepherd for waking them up.}          \\ \hline
\makecell[l]{\textbf{False Negative (FN):}\\
Reality: A wolf threatened. \\
Shepherd said: "No wolf." \\
Outcome: The wolf ate all the sheep.}          & 
\makecell[l]{\textbf{True Negative (TN):}\\
Reality: No wolf threatened. \\
Shepherd said: "No wolf." \\
Outcome: Everyone is fine.  }                              \\ \hline
\end{tabular}
\end{table}
% Emacs 25.2.2 (Org mode 8.2.10)
\end{document}