% Created 2019-11-24 Sun 13:28
\documentclass[a4paper]{article}
\usepackage[utf8]{inputenc}
\usepackage[T1]{fontenc}
\usepackage{fixltx2e}
\usepackage{graphicx}
\usepackage{longtable}
\usepackage{float}
\usepackage{wrapfig}
\usepackage{rotating}
\usepackage[normalem]{ulem}
\usepackage{amsmath}
\usepackage{textcomp}
\usepackage{marvosym}
\usepackage{wasysym}
\usepackage{amssymb}
\usepackage{hyperref}
\tolerance=1000
\usepackage[margin=3cm]{geometry} 	   % Choose your margin here.
\usepackage{tikz,pgfplots}
\usetikzlibrary{calc,patterns,arrows,decorations.pathmorphing,decorations.markings}
\usepackage{array,makecell,multirow}
\pgfplotsset{width=16cm,height=6cm, compat=1.8}
\usepackage{amsmath,mathtools,amssymb,mathrsfs}
\setcounter{secnumdepth}{1}
\author{nybo}
\date{\today}
\title{readme}
\hypersetup{
  pdfkeywords={},
  pdfsubject={},
  pdfcreator={Emacs 25.2.2 (Org mode 8.2.10)}}
\begin{document}

\maketitle
\tableofcontents

Things i keep forgetting
\includegraphics[height=0.2\textwidth]{./imgs/5332.png}

\section{Norms}
\label{sec-1}
\subsection*{def}
\label{sec-1-1}
\begin{enumerate}
\item Positivity  $\|x\| \geq 0$
\item Positive definiteness $\|x\|=0 \Longleftrightarrow x=0$
\item Homogeneity $\|\alpha x\|=|\alpha|\|x\|$ for arbitrary scalar $\alpha$
\item Triangle inequality $\|x+y\| \leq\|x\|+\|y\|$
\end{enumerate}
Note: not sure if this holds for ever norm
\subsection*{The different matrix ones (Norms on $A \in \mathbb{R}^{m \times n}$)}
\label{sec-1-2}
$$
\|A\|_{1}=\max _{1 \leq j \leq n} \sum_{i=1}^{m}\left|a_{i j}\right|
$$

$$
\|A\|_{2}=\sqrt{\lambda_{\max }\left(A^{T} A\right)}
$$

Frobenius norm, sometimes also called the Euclidean norm

$$
\|\mathrm{A}\|_{F} \equiv \sqrt{\sum_{i=1}^{m} \sum_{j=1}^{n}\left|a_{i j}\right|^{2}}
$$

$$
\|A\|_{\infty}=\max _{1 \leq i \leq m} \sum_{j=1}^{n}\left|a_{i j}\right|
$$
\subsection*{Vector norms (Norms on $\mathbb{R}^{n}$)}
\label{sec-1-3}

$$|\mathbf{x}|_{p} \equiv\left(\sum_{i}\left|x_{i}\right|^{p}\right)^{1 / p}$$

special case:

$$|\mathbf{x}|_{\infty} \equiv \max _{i}\left|x_{i}\right|$$



Tips: pretty sure think $\|\cdot\|$ usually just refers to the 2-norm.


\subsection*{Scalar norm (Norm on $C[a, b]$)}
\label{sec-1-4}

\begin{equation}
\left.\begin{array}{l}
{\|f\|_{p}=\left(\int_{a}^{b}|f(\tau)|^{p} d \tau\right)^{\frac{1}{p}}, \quad p \in[1, \infty]} \\ 
{\|f\|_{\infty}=\displaystyle\sup _{\scriptscriptstyle a \leq t \leq b}|f(t)| }\end{array} \quad,
\right\} \quad \mathscr{L}_{p}-\text { norms }
\end{equation}

$$
C[0, \infty), \mathscr{L}_{p} \text{ is a Banach space}
$$

$f \in \mathscr{L}_{p} \Leftrightarrow\|f\|_{p}$ is bounded, i.e. $\exists c:\|f\|_{p} \leq c$
\section{Tensor rank}
\label{sec-2}
\begin{table}[h]
\begin{tabular}{cl}
rank                & object   \\
\hline
0                   & scalar   \\
1                   & vector  \\
2                   & matrix (/Dyad)  \\
$\geq 3$ & tensor        
\end{tabular}
\end{table}
Also sometimes  triad, tetrad are used to refer to tensors of 
rank 3 and 4 respectivly. Some refer to the rank of a tensor as
its order or its degree.




\section{SVD}
\label{sec-3}
Example: 
$A=\left[\begin{array}{lll}{0} & {1} & {1} \\ {\sqrt{2}} & {2} & {0} \\ {0} & {1} & {1}\end{array}\right]$

The SVD is defined as
$$
A=P \Sigma Q^{T}
$$

\subsection*{Method}
\label{sec-3-1}
Computing:
$$
A A^{T}=\left[\begin{array}{lll}{2} & {2} & {2} \\ {2} & {6} & {2} \\ {2} & {2} & {2}\end{array}\right]
$$

\begin{equation}
\begin{aligned}
-\lambda^{3}+10 \lambda^{2}-16 \lambda &=-\lambda\left(\lambda^{2}-10 \lambda+16\right) \\
 &=-\lambda(\lambda-8)(\lambda-2) 
\end{aligned}
\end{equation}


Eigenvals of $A A^{T}$ are $\lambda=8, \lambda=2, \lambda=0$, thus the singular values
 are $\sigma_{1}=2 \sqrt{2}, \sigma_{2}=\sqrt{2}\left(\text { and } \sigma_{3}=0\right)$.


Giving out the matrix
$$
\Sigma= 0_{3x3}+\sigma=
\left[\begin{array}{ccc}{2 \sqrt{2}} & {0} & {0} \\ {0} & {\sqrt{2}} & {0} \\ {0} & {0} & {0}\end{array}\right]
$$


Finding the eigenvectors $(A-\lambda I) \mathbf{x}=\mathbf{0}$, we get respetivly to the egienvector
 who is described before: $p_{1}=\left(\frac{1}{\sqrt{6}}, \frac{2}{\sqrt{6}}, \frac{1}{\sqrt{6}}\right)$,
 $p_{2}=\left(-\frac{1}{\sqrt{3}}, \frac{1}{\sqrt{3}},-\frac{1}{\sqrt{3}}\right)$ and 
$p_{3}=\left(\frac{1}{\sqrt{2}}, 0,-\frac{1}{\sqrt{2}}\right)$ (note: normalised vectors).

Yeilding

$$
P= \left[  {p_1}^T {p_2}^T {p_3}^T     \right]
=\left[\begin{array}{ccc}{\frac{1}{\sqrt{6}}} & {-\frac{1}{\sqrt{3}}} & {\frac{1}{\sqrt{2}}} \\ {\frac{2}{\sqrt{6}}} & {\frac{1}{\sqrt{3}}} & {0} \\ {\frac{1}{\sqrt{6}}} & {-\frac{1}{\sqrt{3}}} & {-\frac{1}{\sqrt{2}}}\end{array}\right]
$$




$$
A^{T} A=\left[\begin{array}{ccc}{2} & {2 \sqrt{2}} & {0} \\ {2 \sqrt{2}} & {6} & {2} \\ {0} & {2} & {2}\end{array}\right]
$$
With the eigenvals $\lambda=8, \lambda=2, \lambda=0$ with eigenvectors
$q_{1}=\left(\frac{1}{\sqrt{6}}, \frac{3}{\sqrt{12}}, \frac{1}{\sqrt{12}}\right), q_{2}=\left(\frac{1}{\sqrt{3}}, 0,-\frac{2}{\sqrt{6}}\right) \text { and } q_{3}=\left(\frac{1}{\sqrt{2}},-\frac{1}{2}, \frac{1}{2}\right)$. (Acually
 can also use the the formula $p_{i}=\frac{1}{\sigma_{i}} A^{T} p_{i}$ to get the various $q_i$.


$$
Q=\left[  {q_1}^T {q_2}^T {q_3}^T     \right]
\left[\begin{array}{ccc}{\frac{1}{\sqrt{6}}} & {\frac{1}{\sqrt{3}}} & {\frac{1}{\sqrt{2}}} \\ {\frac{3}{\sqrt{12}}} & {0} & {-\frac{1}{2}} \\ {\frac{1}{\sqrt{12}}} & {-\frac{2}{\sqrt{6}}} & {\frac{1}{2}}\end{array}\right]
$$


We have then the SVD defined as
$$
A=P \Sigma Q^{T}
$$

\section{Covar}
\label{sec-4}
% Emacs 25.2.2 (Org mode 8.2.10)
\end{document}